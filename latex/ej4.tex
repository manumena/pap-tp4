\section{Ejercicio 4}

Peso asignado: 6.

\subsection{Introducción}

En este problema se pedía un algoritmo que dada una permutación $P$ de números
entre 1 y $N$, devolviera la cantidad de grafos torneos de $N$ nodos que fueran
isomorfos a sí mismos luego de haber aplicado $P$ sobre el conjunto de vértices.
Dado que el resultado podía llegar a tomar valores muy grandes la respuesta
se devuelve módulo $10^9 + 7$. Por último, la complejidad temporal pedida era
$\ord(N^2 \times \log N)$ pudiendo esta ser mejorada a $\ord(N \times \log N)$.

\subsection{Solución}

\subsubsection{Permutación como conjunto de ciclos}

La solución propuesta utiliza fuertemente el hecho de que una permutación $P$ de
números entre 1 y $N$ puede ser representada mediante un conjunto de ciclos
dirigidos donde cada arco $(u, v)$ corresponde a $(u, P(u))$ donde $P(u) = v$.

\newcolumntype{M}[1]{>{\centering\arraybackslash}m{#1}}

\begin{table}[H]
    \caption{Ejemplo de permutación $P$ con $N = 4$.} \label{ej4:table:perm}
    \centering
    \begin{tabular}{|c|M{2em}|M{2em}|M{2em}|M{2em}|}
        \hline
        $x$ & 1 & 2 & 3 & 4 \\
        \hline
        $P(x)$ & 2 & 3 & 1 & 4 \\
        \hline
    \end{tabular}
\end{table}

\begin{figure}[H]
    \caption{Representación de la permutación definida en el Cuadro
    \ref{ej4:table:perm} mediante ciclos dirigidos.} \label{ej4:fig:perm}
    \centering
    \begin{tikzpicture}
        \SetGraphUnit{2}
        \GraphInit[vstyle=Normal]
        \tikzset{EdgeStyle/.style={->}}
        \Vertex{1}
        \EA(1){2}
        \SO(1){3}
        \EA(3){4}
        \Edge[style={bend left=30}](1)(2)
        \Edge[style={bend left=30}](2)(3)
        \Edge[style={bend left=30}](3)(1)
        \Loop[dist=4em,dir=SOEA](4)
    \end{tikzpicture}
\end{figure}

A raíz de esto lo primero que haremos será estudiar las permutaciones formadas
por únicamente un ciclo.

\subsubsection{Permutaciones de un ciclo} \label{ej4:sec:unciclo}

Lo que interesa analizar es el número de grafos torneos de tamaño $N$ que luego
de aplicarles la permutación $P$ son isomorfos al grafo torneo original. Un
grafo torneo es un grafo completo donde todas las aristas tienen asignada una
dirección. Formalmente, se desea contar los grafos torneo $G$ que cumplen
$(\forall (v, w) \in E(G)) (P(v), P(w)) \in E(P(G))$.

Una condición necesaria para que se cumpla esto, es que todos los elementos
pertenecientes al mismo ciclo de una permutación posean el mismo grado de
entrada y salida entre sí en el grafo torneo.

Supongamos que esto no es cierto, entonces debería existir un par de vértices
$v_0, w_0 (v_0 \neq w_0)$ con distinto grado de salida pertenecientes a un grafo
torneo isomorfo a su permutación de un ciclo. Tanto $v_0$ y $w_0$ tendrán un
predecesor en la permutación vista como ciclo. Llamemos $v_1$ el vértice tal que
$P(v_1) = v_0$ y $w_1$ el que cumple $P(w_1) = w_0$. Para que $G$ sea isomorfo a
su permutación, como $v_1$ toma el lugar de $v_0$, tiene que tener el mismo
grado de salida, lo mismo con $w_1$ y $w_0$. Este razonamiento se puede aplicar
a los predecesores de $v_1$ y $w_1$, forzando a que existan $v_2$ y $w_2$
manteniendo los correspondientes grados de salida. Si se sigue aplicando esta
lógica, llegará un momento donde el predecesor de $v_k$ será $w_0$ o bien el de
$w_k$ será $v_0$. Esto implica que $P(w_0) = v_k$ o $P(v_0) = w_k$, lo cual no
es posible, ya que teniendo distinto grado de salida el isomorfismo no sería
valido. La demostración es análoga para el grado de entrada.

Habiendo dicho esto, demostraremos que para permutaciones que forman un ciclo
par, no existe un grafo torneo que sea isomorfo al aplicarle la misma. Si el
ciclo es par, el grafo torneo correspondiente tiene un número par de nodos.
Supongamos que es posible asignarle el mismo grado de salida a todos los nodos,
para ello podemos utilizar la siguiente igualdad: $\sum_{v \in V(G)} d_{out}(v)
= \#E(G)$. Asumamos que el grado de salida es $k$ para todos los vértices,
entonces tenemos: $N \times k = \#E(G) = \frac{N \times (N - 1)}{2}$. Dividiendo
por $N$ de ambos lados nos queda $k = \frac{N - 1}{2}$ que como $N$ es par, $N -
1$ es impar, por lo tanto no divisible por 2. Queda así demostrado que si el
tamaño del ciclo de la permutación es par, no es posible asignarles el mismo
grado a todos los nodos lo cual lleva a que no se cumpla la condición necesaria
enunciada anteriormente.

Para los ciclos impares primero veremos que es posible tener todos los nodos con
el mismo grado de salida, luego demostraremos la cantidad de grafos torneos que
cumplen lo pedido en función del tamaño del ciclo.

Aplicando la misma igualdad que en el párrafo anterior tenemos $N \times k =
\#E(G) = \frac{N \times (N - 1)}{2}$ donde al dividir por $N$ de ambos lados nos
queda $k = \frac{N - 1}{2}$ que ahora teniendo $N$ impar, $N - 1$ es par, por lo
tanto divisible por 2. Así llegamos a que no sólo es posible asignarle el mismo
grado de salida a cada nodo si no que además el mismo vale $\frac{N - 1}{2}$.

Queda entonces ver cuántos grafos torneo cumplen ser isomorfos con su
permutación siendo esta un ciclo impar. Para esto se utilizará un argumento
similar al utilizado para probar que no podían existir dos vértices con distinto
grado. Sabemos que todos los nodos tienen $\frac{N - 1}{2}$ arcos salientes y
por ende $\frac{N - 1}{2}$ arcos entrantes, lo que habría que demostrar es cómo
asignarlos en cada vértice de forma tal que se cumpla el isomorfismo.

A continuación demostraremos que únicamente los grafos torneos donde todos los
vértices se conectan de la misma forma hacia sus $\frac{N - 1}{2}$ sucesores del
ciclo de la permutación forman un isomorfismo válido. Supongamos que esto no es
cierto, por lo tanto existe un par de vértices $v_0, w_0 (v_0 \neq w_0)$ que se
conecta de forma distinta a los correspondientes $\frac{N - 1}{2}$ sucesores del
ciclo de la permutación. Aquí es donde aplicamos el mismo razonamiento que en la
demostración de los grados de los vértices. Existen $v_1$ y $w_1$ tal que
$P(v_1) = v_0$ y $P(w_1) = w_0$ los cuales al tomar el lugar de $v_0$ y $w_0$ en
el grafo torneo permutado deberán estar conectados de la misma forma a sus
$\frac{N - 1}{2}$ sucesores. Se realiza el proceso de ir repitiendo este
razonamiento en los predecesores hasta que se tiene $P(v_0) = w_k$ o $P(w_0) =
v_k$ que como difieren en la forma de saltar a sus sucesores no permite un
isomorfismo válido.

Habiendo probado esto, calcular el número de grafos torneo posibles para una
permutación correspondiente a un ciclo impar de longitud $N$ es equivalente a
contar la cantidad de formas en las que un vértice puede saltar a sus $\frac{N -
1}{2}$ sucesores. Cada arco puede ser tanto saliente como entrante, y se tienen $\frac{N -
1}{2}$ a definir, por lo tanto un total de $2^{\frac{N -1}{2}}$ combinaciones
posibles.

\subsubsection{Permutaciones entre ciclos} \label{ej4:sec:paresciclos}

Como se ve en la Figura \ref{ej4:fig:perm}, una permutación puede estar
representada por más de un ciclo. Vimos cómo calcular el número de combinaciones
que brinda un ciclo en particular ahora es necesario ver las combinaciones
posibles entre todos los que componen la permutación.

Para esto estudiaremos las combinaciones posibles entre un par de ciclos, ya que
veremos las combinaciones que suman los otros ciclos son independientes entre
sí. Teniendo dos ciclos impares $C_0$ y $C_1$ para obtener un grafo torneo deben
existir todas las aristas entre ellos con una dirección asignada. A su vez,
queremos que al aplicar la correspondiente permutación, por cada arco $(v, u)$
de $C_0$ a $C_1$ exista $(P(v), P(u)) \in E(P(G))$ al igual que con los
arcos $(u, v)$ de $C_1$ a $C_0$.

Tomando un $v_0$ cualquiera de $C_0$ hay $N$ arcos a asignar contra $C_1$. Una vez
que estos están asignados, sabemos que existe $v_1$ tal que $P(v_1) = v_0$, con
lo cual deseamos que al tomar su lugar valga el isomorfismo. Para ello, la forma
de asignar las direcciones de $v_1$ contra $C_1$ debe ser tal que al aplicar la
permutación a cada uno de sus arcos corresponda con las asignaciones de $v_0$
contra $C_1$ en el grafo original. Esto se puede aplicar al predecesor $v_2$ con
$P(v_2) = v_1$ y repetirlo hasta dar toda la vuelta formando entonces un
isomorfismo válido bajo la restricción de que cada nodo se conecte contra $C_1$
utilizando las mismas direcciones contra los nodos que al aplicarles la
permutación serán equivalentes a las asignaciones de su sucesor.

Si se decidiera asignar las direcciones sin respetar este criterio, para algún
vértice ocurrirá que luego de la permutación las asignaciones serán distintas a
las del grafo original, por lo tanto no corresponderá a un isomorfismo válido.

Mediante este método, el número de combinaciones entre un par de ciclos impares
es la cantidad de formas de asignar direcciones del del mayor al menor. Se
tienen dos direcciones posibles y $\min(\#V(C_0), \#V(C_1))$ arcos, lo cual
corresponde a un total de $2^{\min(\#V(C_0), \#V(C_1))}$ combinaciones por cada
par de ciclos.

El hecho de que cada par de ciclos agregue independientemente del resto esta
cantidad de combinaciones posibles se debe a que la forma en la que un ciclo
asigna sus arcos para sí mismo así como contra el resto de los ciclos no afecta
la elección de arcos contra el par elegido en particular. Lo único que se asume
es que se comienza con dos ciclos impares cuya asignación de arcos para sí y
contra el resto es válida con respecto al isomorfismo buscado.

\subsubsection{Algoritmo \emph{Divide \& Conquer}}

El algoritmo desarrollado para contar el total de combinaciones posibles hace
uso de la técnica algorítmica \emph{Divide \& Conquer} donde dado un arreglo $C$ con
los tamaños de los ciclos correspondientes a la permutación realiza los
siguientes pasos:

\begin{itemize}
	\item Si $|C| = 1$, si el tamaño del ciclo es par, devuelvo 0, caso contrario
    $2^{\frac{N -1}{2}}$ con $N$ siendo el tamaño del ciclo.
    \item Si $|C| > 1$ entonces llamo el algoritmo sobre el intervalo $\left[0,
    \frac{|C|}{2} \right)$ y $\left[\frac{|C|}{2}, |C| \right)$, multiplico
    ambos resultados y los multiplico a su vez por el número de combinaciones
    entre los ciclos de ambas mitades.
\end{itemize}

A continuación se presenta el pseudocódigo del algoritmo que aplica \emph{Divide
\& Conquer}:

\begin{algorithm}[H]
    \caption{Número de grafos torneos isomorfos a su permutación $P$}
    \Input{Arreglo $C$ con las longitudes de los ciclos de $P$ e índices $i$ y
    $j$,}
    \Output{Devuelve el número de grafos torneos isomorfos a sí mismos con la
    permutación $P$ módulo $10^9 + 7$.}

    \eIf{$j - i == 0$} {
        \Return{1} \;
    }
    {
        \eIf{$j - i == 1$} {
            \Return{\emph{$0$ si el ciclo es impar, $2^{\frac{C[i] - 1}{2}}$ caso
            contrario}} \;
        }
        {
            \emph{primeraMitad} $\gets$
                \emph{calcular y guardar combinaciones posibles para los ciclos
                en el intervalo $\left[ i, \frac{i + j}{2} \right)$} \;
            \emph{segundaMitad} $\gets$
                \emph{calcular y guardar combinaciones posibles para los ciclos
                en el intervalo $\left[ \frac{i + j}{2}, j \right)$} \;
            \emph{combinacionesPosibles} $\gets$
                \emph{(primeraMitad $\times$ segundaMitad)
                \% $(10^9 + 7)$} \;
            \emph{combinacionesPosibles} $\gets$
                \emph{(combinacionesPosibles $\times$
                número de combinaciones entre ciclos de ambas mitades)
                \% $(10^9 + 7)$} \;

            \Return{\emph{combinacionesPosibles}} \;
        }
    }
\end{algorithm}

El algoritmo anterior recibe como entrada un arreglo $C$ con las longitudes de
los ciclos de $P$, el mismo se genera mediante el siguiente algoritmo:

\begin{algorithm}[H]
    \caption{Longitudes de ciclos de permutación $P$}
    \Input{Arreglo con la permutación $P$.}
    \Output{Devuelve un arreglo $C$ con las longitudes de los ciclos de la
    permutación $P$.}

    \emph{longitudesCiclos} $\gets$ \emph{lista vacía} \;
	\For{\emph{i} \textbf{en} $\left[ 0, |P| \right)$} {
        \If{\emph{$P[i]$ no fue visitado}} {
            \emph{longitudCiclo} $\gets$ $1$ \;
            \emph{j} $\gets$ \emph{i} \;
            \emph{marcar a $P[j]$ como visitado} \;
            \While{$P[j] \neq j + 1$} {
                \emph{longitudCiclo} $\gets$ \emph{longitudCiclo} + 1 \;
                \emph{j} $\gets$ $P[j]$ \;
                \emph{marcar a $P[j]$ como visitado} \;
            }
            \emph{longitudesCiclos}\textbf{.agregar}(\emph{longitudCiclo}) \;
        }
    }
    \Return{\emph{longitudesCiclos}} \;
\end{algorithm}

Por último, el algoritmo utilizado para calcular las combinaciones entre ciclos
de dos mitades:

\begin{algorithm}[H]
    \caption{Cantidad de combinaciones entre los ciclos de la primer y segunda
    mitad}
    \Input{Arreglo $C$ con longitudes de ciclos e índices $i$ y $j$.}
    \Output{Cantidad de combinaciones posibles entre los ciclos de la primera y
    la segunda mitad.}

    \emph{totalPermutaciones} $\gets$ 1 \;
    \emph{ciclosPrimeraMitad} $\gets$ \emph{diccionario donde las llaves son los
    tamaños de los ciclos y los valores la cantidad existente con ese tamaño
    para la primer mitad} \;
    \emph{ciclosSegundaMitad} $\gets$ \emph{diccionario donde las llaves son los
    tamaños de los ciclos y los valores la cantidad existente con ese tamaño
    para la segunda mitad} \;

    \ForEach{\emph{clavePrimeraMitad} \textbf{en} \emph{ciclosPrimeraMitad}} {
        \ForEach{\emph{claveSegundaMitad} \textbf{en} \emph{ciclosSegundaMitad}} {
            \emph{cicloMásChico} $\gets$
                \emph{mínimo entre clavePrimeraMitad y claveSegundaMitad} \;
            \emph{permutaciones} $\gets$
                $(2^{\emph{cicloMásChico}})^{\emph{ciclosSegundaMitad[claveSegundaMitad]}}$ \;
            \emph{permutaciones} $\gets$
            $\emph{permutaciones}^{ciclosPrimeraMitad[clavePrimeraMitad]}$ \;
            \emph{totalPermutaciones} $\gets$
            (\emph{totalPermutaciones} $\times$ \emph{permutaciones}
            \emph{\% $(10^9 + 7)$} \;
        }
    }
    \Return{\emph{totalPermutaciones}} \;
\end{algorithm}

\subsubsection{Correctitud}

Para ver que el algoritmo es correcto es posible realizar inducción sobre el
tamaño del arreglo $C$ con las longitudes de ciclos de la permutación $P$. El
caso base sería arreglo de un elemento donde vemos que es correcto ya que hace
la cuenta desarrollada en la Sección \ref{ej4:sec:unciclo} para calcular el
número de combinaciones con un único ciclo. Para el paso inductivo, teniendo
como hipótesis inductiva que el algoritmo funciona para arreglos $C$ de longitud
$N - 1$ es posible asumir que las llamadas a las dos mitades calculan de forma
correcta las combinaciones posibles en cada una de ellas. Por lo tanto, una vez
que se calculan todas las posibles combinaciones entre ellas siguiendo lo
descrito en la Sección \ref{ej4:sec:paresciclos} el algoritmo es correcto para
$N$.

\subsection{Complejidad teórica}

El algoritmo desarrollado tiene una complejidad temporal $\ord(N \times \log
N)$. Para probar esto estudiaremos la resolución de a partes.

Lo primero que se hace es transformar la permutación $P$ en el arreglo con
longitudes de ciclos $C$. Esto tiene un costo $\ord(N)$ ya que se iteran los
elementos de $P$ y al marcarlos como visitados no se vuelven a recorrer los
ciclos que ya se agregaron.

Para demostrar la complejidad del \emph{Divide \& Conquer} primero estudiaremos
la complejidad de calcular las combinaciones posibles entre dos mitades de
ciclos.

En esta rutina primero se cuenta la cantidad de ciclos de cierta
longitud que hay en cada mitad. Esto en el peor caso es $\ord(N)$, que sería
cuando todos los ciclos son de un único elemento. A esto se le suma el recorrer
ambos mapas y generar el producto correspondientes. Cada mapa tiene a lo sumo
tamaño $\sqrt{N}$ ya que si hubiera más de $\sqrt{N}$ tamaños de ciclos entonces
la suma de los mismos sería distinta a $N$. Por lo tanto, la iteración de ambos
mapas tiene un costo $\ord(\sqrt{N} \times \sqrt{N}) = \ord(N)$. El costo de
calcular las potencias módulo $10^9 + 7$ es constante puesto que siempre se está
operando con valores módulo $10^9 + 7$, por lo tanto la operación es sobre una
entrada que está acotado por una constante.

Así demostramos entonces que calcular las combinaciones entre ambas mitades
tiene costo lineal. Luego, dado que el \emph{Divide \& Conquer} parte a
la mitad hasta llegar a elementos de tamaño uno, el mismo llama $\ord(\log{N})$
veces el paso de \emph{merge} donde se realiza la llamada con costo lineal más
los productos que ya dijimos eran constantes.

Por lo tanto, tenemos $\ord(N \times \log N)$ como complejidad temporal.

\subsection{Casos de prueba}

Para probar el correcto funcionamiento del algoritmo se realizaron pruebas a
distinto nivel:

\begin{itemize}
    \item Pruebas sobre la función de potencia con módulo
    \item Pruebas sobre la función que cuenta los ciclos en $P$
    \item Pruebas sobre la función que cuenta la cantidad de combinaciones entre
    dos mitades de ciclos
    \item Pruebas sobre la función que cuenta el total de torneos isomorfos a su
    permutación $P$
\end{itemize}

Los primeros tres tipos de pruebas fueron para asegurar el correcto
funcionamiento con números que estuvieran por debajo y encima del valor $10^9 +
7$, así como ver que cada módulo cumpliera correctamente su función para
distintos tipos de permutaciones (de un único ciclos, varios ciclos, ciclos
formados por un único elemento).

La última prueba verifica la funcionalidad final del algoritmo, utilizando primero
entradas pequeñas controladas con distintos tipos de permutaciones:

\begin{itemize}
    \item Un ciclo de un elemento
    \item Un ciclo de tres elementos
    \item Dos ciclos
    \item Tres ciclos
    \item Cuatro ciclos de un elemento
    \item Ciclos pares
\end{itemize}

Y por último buscando probar resultados que excedieran el valor $10^9 +
7$ por un lado, y por el otro con permutaciones de tamaño $10^5$ para corroborar
que el algoritmo pudiera manejar ese tamaño de entrada.
