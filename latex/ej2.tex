\section{Ejercicio 2}

Peso asignado: 8.

\subsection{Introducción}

Se tiene un arreglo de $N$ enteros y se desea ordenarlo en BigTalk. El
lenguaje es capaz de permutar aleatoriamente los últimos $N - i$ númneros del
arreglo, siendo todas las posibles permutaciones equiprobables.

El objetivo del problema es obtener ese arreglo y calcular cuánto tardaría en
ordenarse con un programa en Bigtalk que haga lo siguiente:

\begin{enumerate}
\item Se comienza con $i = 0$ y el arreglo $A$ de tamaño $N$ dado en la
entrada
\item Se permuta al azarel arreglo $A[i..N)$
\item Mientras que $i$ sea menor a $N$ y el menor elemento de $A[i..N)$ sea
$A[i]$ se incrementa $i$ en 1
\item Si $i < N$ se vuelve al paso 2
\end{enumerate}

Para calcular cuánto tarda el algoritmo se debe contar la cantidad de veces
que se repite el paso 2 ya que el tiempo consumido en el resto de los pasos es
despreciable, por lo que se hace enfoque en contar cuántas veces se permuta un
sufijo de $A$ aleatoriamente. Debido al azar presente, la cantidad de
repeticiones necesarias para ordenar el arreglo no es posible de determinar
con exactitud, por lo que se pide calcular la esperanza.

\subsection{Solución}

\subsubsection{Explicación}

Se quiere calcular la esperanza de la cantidad de veces que se repite el paso
2 al correr el algoritmo en BigTalk previamente mencionado. Sea $P$ la
cantidad de veces que se repite el paso 2 y $P_i$ la cantidad de veces que se
repite el paso 2 en la iteración $i$ con $0 \leq i < N$, puede decirse que
$P = \sum_{i = 0}^{N}{P_i}$. En consecuencia,
$E(P) = E(\sum_{i = 0}^{N}{P_i})$, siendo $E(P)$ la esperanza de $P$.

Debido a la linealidad de la esperanza se sabe que
$E(X_1 + ... + X_n) = E(X_1) + ... = E(X_n)$ por lo que
$E(\sum_{i = 0}^{N}{P_i}) = \sum_{i = 0}^{N}{E(P_i)}$.

$P_i$ es la cantidad de veces que se repite el paso 2 en la iteración $i$, que
es lo mismo que la cantidad de veces que se permutan aleatoriamente los
valores de $A[i..N)$. En cada iteración, la cantidad de veces que se repite el
paso 2 será la cantidad de veces que se permute el subarreglo hasta que el
mínimo elemento quede en la posición $i$. La probabilidad de que esto ocurra
es igual a la cantidad de veces que se repite el mínimo en $A[i..N)$ dividido
la cantidad de elemento del subarreglo, es decir
$p_i = \frac{\#min(A[i..N))}{N - i}$, ya que las permutaciones son
aleatoriamente distribuídas de manera uniforme; o sea que todas las
permutaciones tienen la misma probabilidad.

Puede verse que $P_0$, es decir en la primer iteración, se comporta como una
variable aleatoria Geométrica. $P_0$ es la cantidad de repeticiones hasta el
primer éxito, en este caso es conseguir que el mínimo valor se sitúe en la
primer posición. Entonces $P_0$ es una Geométrica de parámetro
$p_0 = \frac{\#min(A)}{N}$. Entonces
$E(P_0) = \frac{1}{p_0} = \frac{N}{\#min(A)}$.

Esto no ocurre con el resto de las iteraciones, ya que existe la probabilidad
de que en la iteración anterior, cuando el minimo del subarreglo actual haya
caído en la posición correspondiente. Es decir que se haya cumplido que el
mínimo de la iteración $i$ caiga en la posición $i$ y que además el siguiente
mínimo haya caído en la $i + 1$. Para abordar el resto de las iteraciones se
define entonces un función de probabilidad puntual que determina cuál es la
probabilidad de que se repita el paso 2 una $x$ cantidad de veces con
$x \geq 0$.

La probabilidad puntual de $A_i$ es entonces $p_{A_i} = p (1 - p)^x$ con
$i \geq 0$ y $p = \frac{\#min(A[i..N))}{N - i}$.

Solamente resta calcular la esperanza de $A_i$ y con eso se tiene la esperanza
total del algoritmo.

Sea $q = (1 - p)$

\[ \begin{aligned}
    E(A_i)  &= \sum_{x = 0}^{\infty}{x p q^x} \\
    		&= p \sum_{x = 0}^{\infty}{x q^x} \\
    		&= p \sum_{x = 0}^{\infty}{\left( \frac{\partial q^{x + 1}}{\partial q} - q^x \right)} \\
    		&= p \left( \sum_{x = 0}^{\infty}{\frac{\partial q^{x + 1}}{\partial q}} - \sum_{x = 0}^{\infty}{q^x} \right) \\
    		&= p \left( \frac{\partial \sum_{x = 0}^{\infty}{q^{x + 1}}}{\partial q} - \sum_{x = 0}^{\infty}{q^x} \right) \\
    		&= p \left( \frac{\partial (q \sum_{x = 0}^{\infty}{q^x})}{\partial q} - \sum_{x = 0}^{\infty}{q^x} \right) \\
    		&= p \left( \frac{\partial \frac{q}{1 - q}}{\partial q} - \frac{1}{1 - q} \right) \\
    		&= p \left( \frac{1}{(1 - q)^2} - \frac{1}{1 - q} \right) \\
    		&= p \left( \frac{1}{(1 - q)^2} - \frac{1 - q}{(1 - q)^2} \right) \\
    		&= \frac{pq}{(1 - q)^2} \\
    		&= \frac{q}{p} \\
    		&= \frac{1 - p}{p} \\
    		&= \frac{1}{p} - 1 \\
\end{aligned} \]

De esta forma queda que $E(A_i) = \frac{N - i}{\#min(A[i..N))} - 1$

Finalmente para calcular la esperanza total del algoritmo se suma la esperanza
de la Geométrica que describe la primer iteración con el resto de las
esperanzas de las iteraciones siguientes. Para ello es necesario calcular
cada esperanza por separado, calculando la cantidad de veces que se repite el
mínimo en el subarreglo.

\subsection{Algoritmo}

El algoritmo es muy sencillo, consiste en un bucle de $N$ iteraciones en donde en cada iteracion se cuenta la cantidad de veces que se repite el elemento de mínimo valor en el subarreglo $A[i..N)$ de manera lineal.

\bigskip

\begin{algorithm}[H]
	\caption{\textit{esperanza}}
	\Input{ Arreglo de enteros $A$ }
	\Output{ Doble $esperanza$ }

	$e \gets$ $A$.tamaño() $/$ cantMin($A$) \;

    \For {$i = 0$ hasta $i = A$.tamaño()} {
		$e \gets e + (A$.tamaño() $- i) /$ cantMin($A[i..N)$) $- 1$ \;
	}

	\Return{e}
\end{algorithm}

\bigskip

\begin{algorithm}[H]
	\caption{\textit{cantMin}}
	\Input{ Arreglo de enteros $A$ }
	\Output{ Entero $cant$ }

	$cant \gets 1$ \;
	$min \gets A[0]$ \;

    \For {$i = 1$ hasta $i = A$.tamaño()} {
    	\If {$A[i] = min$} {
    		$cant++$ \;
    	}
    	\If {$A[i] < min$} {
			$min \gets A[i]$ \;
    		$cant \gets 1$ \;
    	}
	}

	\Return{cant}
\end{algorithm}

\bigskip
