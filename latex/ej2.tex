\section{Ejercicio 2}

Peso asignado: 8.

\subsection{Introducción}

Se tiene un arreglo de $N$ enteros y se desea ordenarlo en BigTalk. El
lenguaje es capaz de permutar aleatoriamente los últimos $N - i$ númneros del
arreglo, siendo todas las posibles permutaciones equiprobables.

El objetivo del problema es obtener ese arreglo y calcular cuánto tardaría en
ordenarse con un programa en Bigtalk que haga lo siguiente:

\begin{enumerate}
\item Se comienza con $i = 0$ y el arreglo $A$ de tamaño $N$ dado en la
entrada
\item Se permuta al azarel arreglo $A[i..N)$
\item Mientras que $i$ sea menor a $N$ y el menor elemento de $A[i..N)$ sea
$A[i]$ se incrementa $i$ en 1
\item Si $i < N$ se vuelve al paso 2
\end{enumerate}

Para calcular cuánto tarda el algoritmo se debe contar la cantidad de veces
que se repite el paso 2 ya que el tiempo consumido en el resto de los pasos es
despreciable, por lo que se hace enfoque en contar cuántas veces se permuta un
sufijo de $A$ aleatoriamente. Debido al azar presente, la cantidad de
repeticiones necesarias para ordenar el arreglo no es posible de determinar
con exactitud, por lo que se pide calcular la esperanza.